%%% File encoding is ISO-8859-1 (also known as Latin-1)
%%% You can use special characters just like ä,ü and ñ

\chapter{Methods}

\section{Agent-based Modeling}
We used agent-based modeling in this project to simulate the effect of different Proof-of-Stake designs in economic centralization. Agent-based modeling simulates how micro-level decisions aggregate and interact to produce a macro-level system phenomena. For example, the micro pattern of the person-to-person spreading of the flu produces a macro pattern in the global landscape. In agent-based modeling, those micro-level interactions are encoded in the agents using simple rules to interact with each other. Instead of choosing completely random partners, agents usually only interact with those who are spatially close to them or connected to them on a social network. System level emergence, such as spread of disease, adoption of new technology, or composition of a community emerge as a result of such an interaction.

In our simulation, we designed the agent to be interacting as part of a network where they stake coins and get rewarded for becoming a validator. An agent's wealth, rules of interaction, and the selection process of validators between interacting agents are three elements that are particularly important in our model. Agents have heterogeneous internal states of wealth and greed preferences. Those internal states impact their interaction rules with other agents. While some rules are fixed for an agent's life, others rules can change in response to external environments, past interactions, expectations of future interactions, or even adopt a counterpart's strategy \cite{CRYPTO:9}. 

\section{Proof-of-Stake Simulation}
For us to begin agent-based modeling, we needed to define the network we want to simulate. Since we wanted an adequate sample size to simulate a network with a lot of activity, we configured the network to start with 10,000 nodes (or agents). We then defined a simulation around this network. We wanted to observe the effects of each Proof-of-Stake protocol on the network's Gini coefficient, so we defined two initial states for the network to begin in.

We wanted to observe what would happen to a network with a fairly even distribution, so our first initial state is normal distribution. In normal distribution, we distributed wealth across the network with a mean of 100 coins and a standard deviation of 10. This distribution with random numbers usually starts off with an initial Gini coefficient of 5\%. We also wanted to know the protocols' effects on an already wealth-centralized network like Bitcoin or Ethereum. Therefore, we have another initial state called "real" distribution. In this state, 80\% of the network starts off with just approximately 10 coins, another 15\% of the network starts off with about 100 coins, and the remaining 5\% start off with 1000 coins. This distribution starts off with a Gini coefficient of slightly below 80\%. We chose this distribution instead of replicating the current wealth distribution in Bitcoin because Bitcoin's wealth distribution was too extreme. We wanted to run simulations where it was still possible to observe a change in the Gini coefficient.

After initializing the nodes in the network, we repeatedly use a defined Proof-of-Stake protocol to determine the validator(s) of the next block and distribute the block reward. The base block reward for all simulations was set at 20 coins. In each Proof-of-Stake system we ran the simulation for 10,000 iterations and calculated the Gini coefficient to observe the long-term effects of each protocol.

Finally, in order to model network behavior, we decided to introduce elements of game theory into our design \cite{CRYPTO:3}. We designated a certain amount of "greed" for each node as a ratio from 0 to 1 where 0 is not at all greedy and 1 means completely greedy. This greed factor helps us determine node behavior as we progress through our simulation. In addition, the number of blocks mined and the node's greed ratio is also adjusted to match the distribution being used, with richer nodes beginning with a higher number of blocks mined and a higher greed ratio in a real distribution.

In our investigation, we evaluated the impact of reward distribution algorithms from different Proof-of-Stake systems (Pure Proof-of-Stake, Delegated Proof-of-Stake, and Extended Proof-of-Stake). We will also be applying principles of game theory in an attempt to design our own Proof of Stake protocol: Raffle Proof-of-Stake.

\subsection{Pure Proof-of-Stake}
Pure Proof-of-Stake (P-PoS) is one of the most straightforward approaches to modeling PoS systems. In P-PoS, players are pseudorandomly selected to propose blocks and vote on new blocks with their probability of selection being directly proportional to the amount of stake they have put into the system. The probability that a node $i$ is chosen as a validator is equal to the proportion of total wealth node $i$ holds, represented by the equation: 
\begin{equation}
P_i = \frac{i_{wealth}}{\sum_{n=1}^{N} n_{wealth}} 
\end{equation}
where N = number of nodes in the network. To model P-PoS, we first created a wealth distribution which takes the total amount of wealth in the network and calculates each user's proportional wealth relative to the total wealth amount. Using this probability distribution, we can then select a given number of validators from the network.

From the set of winners, the reward is evenly split from the prize pool by the number of total winners that were selected. Each winner then adds this reward to their personal wealth, and we recalculate the wealth distribution of the nodes before repeating the selection process.

The security of this system is based on the assumption that a majority of the verifiers are honest. Since the selection of block validators is conducted pseudorandomly, we assume that a majority of the network is honest to ensure that bad blocks are not validated. This system comes with some pros and cons based on this assumption. One benefit of this is that a small number of malicious users have a smaller percent chance of having a harmful influence on the network. This is because there is a small chance of these users being selected if they are a minority amount relative to the size of the network. In other PoS systems, it may be that a small set of users is in charge of selecting winners, which creates the risk of this small group being malicious. Pure Proof-of-Stake also assumes an honest majority because if a majority of stakers are acting maliciously, the value of the currency would ultimately decrease and those malicious users would also be losing money, which would disincentivize them from acting as so. A downside to this approach is that there is still a chance of malicious users being selected since the selection is entirely random. If a dishonest user were to obtain a large amount of stake, they could increase their chance of being selected and harm the network.

\subsection{Delegated Proof-of-Stake}
Delegated Proof-of-Stake is an abstraction on Proof-of-Stake where users stake to a pool rather than individually staking tokens to validate transactions and propose new blocks. Pools are then voted upon based on their stake value to then validate the transactions. This pool would then collect the block reward and distribute the winnings to its constituent stakers after keeping a cut. The reason behind using D-PoS instead of PoS is the benefits it has in terms of speed and scalability. There are many variations of D-PoS using different parameters, but in our simulations, the following D-PoS was used.

The ecosystem was set up with 10,000 participants. Network participants started with some random wealth which was generated based on a distribution. For our simulations we used both normal distributions, and real distributions in order to test the level of centralization in a network when it starts off balanced, and when it starts off already centralized. Following the initial deposit of wealth to each network participant, all network participants are considered to be pool operators. Every time-step, the participants decide which pool to allocate their stake toward.

\textbf{Decision criteria:}
\begin{itemize}
\item Number of blocks mined (proportional to number of blocks mined out of total)
\item Greed factor determines how much the miners take as a cut
\item Popularity regulates how likely stakers are to stake with the same pool again
\end{itemize}

For any delegate $d \in D$, where $D$ is the set of all delegates in the network, the probability mass function that any specific staker stakes with that specific delegate is represented by the following formula. 
\begin{equation}
p(d) = \frac{\mbox{numBlocksMined}(d) \times \mbox{popularity}(d)}{\sum_{i \in D} \mbox{numBlocksMined}(i) \times \mbox{popularity}(i)} 
\end{equation}
The popularity of a pool is determined by the following update formula where $n$ is a time step, $d \in D$ where $D$ is the set of all delegators, and $C$ is the set of all constituent nodes who delegated with delegator $d$ at time step $n$.
\begin{equation}
\mbox{popularity}(d)_{n+1} = \mbox{popularity}(d)_n * (0.995)^{\sum_{c \in C}1_{\{greed(d) > greed(c)\}}}
\end{equation}
Following the staking decisions, the D-PoS algorithm decides which pool to select as the next block creator\slash validator. It makes this decision through pseudorandom selection where the probability of being selected is proportional to that pool's percentage of total tokens staked. Once the block creator is chosen, the 20 coin block reward is distributed. The pool operator is able to decide how much of the reward they keep for themselves and how much gets distributed back to the constituents based on the greed factor. This process is then repeated with network participants giving tokens to a pool, a pool getting selected, and the rewards being distributed.

\subsection{Extended Proof-of-Stake}
The next type of Proof-of-Stake that we observed is Extended Proof-of-Stake (E-PoS). Although E-PoS has not been implemented in a real-world cryptocurrency, the concept was developed and we wanted to test it's proposed functionality \cite{CRYPTO:11}.

In E-PoS, the total transaction cost (sum of all transaction fees in a block) is recorded and maintained.
The first step in determining the validators for the designated blocks in E-PoS is to order all the blocks that need to be validated in decreasing order of transaction cost. Next, validators are given the option to place a bid for the block they want to mine. However, in order to become a potential candidate to validate a block, a node needs to have an account balance greater than the total transaction cost of the block. So, if a node does not contain enough funds, they cannot place a bid to potentially validate a block. After a certain amount of time, the network asks for a confirmation from each node that placed a bid for each block. The nodes first send a portion of their balance (equal to the transaction cost of the block) and some extra "commitment", which both get locked up. Commitment is the portion of the remaining balance of each node that is also sent with the transaction cost associated with being a potential candidate to validate a block. However, if no node contains a minimum amount of wealth that exceeds the total transaction cost of a block, the threshold for becoming a candidate for the block is reduced to allow the sender/receiver to validate the block themselves.

After all bids have been placed, the network runs an algorithm to determine which person to select as the validator for each block.
\spacing{1}
\begin{enumerate}
\item Find the node(s) who committed the most amount from their balance. For example, let's say that the total transaction cost of a block is 50 coins. Let's also assume that node A has 100 coins, while node B has 1000 coins each. Thus, each node is first required to commit 50 coins, leaving node A with 50 coins and node B with 950 coins. If node A then commits the remaining coins (50 coins), and node B commits (200 coins), node A will be chosen because node A committed 100\% of their remaining balance (50\slash50), while node B committed about 21\% (200\slash950). Thus, the node with the most commitment will be chosen.
\item Now, if two or more nodes commit the same amount of coins, then the network checks to see the amount of blocks each node has validated in the past (this would mean the network has to keep track of the number of blocks each node has validated). The network chooses the node that has validated the least amount of blocks. For example, if node A has 5 blocks validated and node B has 10 blocks validated, node A will be chosen.
\item Finally, if two or more nodes have mined the same number of blocks in the blockchain's history, then the node with the higher balance (more coins) is chosen to be the validator. So, from our first example, if node A had 100 coins and node B had 1000 (and we assume that they commit 100\% and both have 5 blocks validated), node B will be chosen since node B's account balance is greater.
\item If in the extreme case that two or more nodes have the exact same amount of coins, then one of those nodes is chosen at random.
\end{enumerate}
\onehalfspacing
However, there were some differences in our design as we could not implement all the features to their exact specifications. First, in the case when no one places a bid to validate a block, we first check to see if there is anyone that placed a bid for a block with a higher total transaction cost and run the selection algorithm on that list of nodes. Second, in the case when there are no alternatives to validate a block, a random node is chosen to become the validator. We implemented these two features because we focused on the selection process\slash incentive layer of a blockchain, so we did not keep track of the sender or receiver of a transaction.

\subsection{Our Own Design: Raffle Proof of Stake}
With the pros and cons of each of these PoS systems in mind, we decided to implement our own incentive layer based on a raffle system, named Raffle Proof-of-Stake (R-PoS). In our design, we used a game theoretic approach to enable nodes to become validators. Every staker can participate in a lottery to determine the next validator by offering up a fixed amount of their stake in exchange for raffle tickets. A node decides to participate in the raffle and pseudorandomly determines how many raffle tickets they wish to purchase weighted by their greed. The 'greedier' a node is, the more likely they are to purchase raffle tickets.

The coins used to purchase the tickets are then added to a "prize pool," in addition to the block reward. Once tickets are staked, a predetermined number of validators are chosen without replacement from the raffle pool. The validators then validate the block and are rewarded with a majority of the prize pool. The distribution ratio is represented by the equation: 
\begin{equation}
R_{winner} = \frac{v^2}{N}
\end{equation}
where $v$ is the number of validators and $N$ is the size of the network. With the remaining coins in the prize pool, another portion will be split out and evenly distributed among those who purchased raffle tickets as a consolation prize. The ratio of this split is determined by the number of people who participated in the raffle using the formula: 
\begin{equation}
R_{participant} = min\left (\frac{N_{participants}}{N_{nonparticipants}}, 0.9 \right )
\end{equation}
So, the more participants in the raffle, the higher the ratio of the consolation prize. Finally, the remaining balance of the prize pool is distributed evenly among the nodes who did not participate in the raffle and chose to receive some interest.

During this period, if a node won the raffle and earned money, their greed would increase by up to 10\%. If a node purchased tickets and won the raffle, but lost money overall, their greed will decrease by up to 30\%. If a node purchased tickets but did not win, they will increase or decrease their greed by a smaller amount (+\slash- 5\%) based on if the consolation prize covered or exceeded the cost of their tickets. The rest of the network who did not participate in the lottery will increase their greed by a miniscule amount (2\%) due to fear of missing out.

The motivation behind this design is that in other PoS protocols, the system incentivises hoarding of wealth. When this happens, it is very difficult for changes in wealth distribution to occur, and we end up seeing the rich becoming richer by holding onto their money. Our design incentivises spending, creating a micro-economy within the network with the raffle system. This allows the currency to shift around the nodes and hopefully establish a balance in wealth distribution. In this system, while it will be easier for the rich to win the raffle by purchasing many tickets, their opportunity cost increases with every ticket they buy. At a certain point, the raffle reward will not be able to cover the total cost of the raffle tickets purchased. Therefore, even if a node is rich, incentives pose an upper limit to the number of raffle tickets purchased and thereby limits their chance of winning. Moreover, our reward distribution method is done evenly, so the reward to a wealthy node will be the same as the reward to a node with a single coin.

Though this seems to cause participants to be disinterested in participating in the lottery, we expect the large prize pool for validators to incentivise participation. Moreover, the consolation prize for participating grows with more participants. So, many will choose to purchase just a single ticket hoping for a large participation rate to get the consolation prize, while other greedier nodes will aim for the much larger validation prize. Overall, there is a social optimum for more nodes to participate. We expect our addition of greedy behavior in the simulation model to be able to reflect these above behaviors, resulting in a more accurate modeling of our Proof-of-Stake design.
