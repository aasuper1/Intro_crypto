%%% File encoding is ISO-8859-1 (also known as Latin-1)
%%% You can use special characters just like �,� and �

\chapter{Discussion}

\section{Conclusion}
The purpose of this study was to evaluate centralization trends in various Proof-of-Stake models and determine whether or not centralization occurs in these systems overtime. By using agent based modeling techniques, we were able to simulate each system with a network of players and observe how each user's wealth changed over time. Based on our observation on the output of these models, we conclude that centralization does occur in Proof-of-Stake systems overtime. We also observe that our proposed model, Raffle Proof-of-Stake (R-PoS) can potentially decrease the rate of centralization, and in an ideal case, mitigate it as a major factor.

Looking at the combined results of the simulated network distribution, E-PoS starts off as the most centralized, but fluctuates down to have the lowest centralization coefficient as wealth becomes more distributed. P-PoS and D-PoS both show a linearly increasing trend of centralization overtime, with D-PoS having the fastest rate of increase. R-PoS has one of the lowest trends of centralization, though it is passed by the E-PoS model after around 8000 iterations. Overall, all of the defined models show a Gini Index greater than zero after starting from a baseline network.

In our simulation of a 'real' distribution, where 80\% of the network starts off with just approximately 10 coins, another 15\% of the network starts off with about 100 coins, and the remaining 5\% start off with 1000 coins, we observed slightly different characteristics of the models. All models start with a Gini Index around 0.80. We observed E-PoS having a logarithmic increase in centralization overtime and P-PoS slight increase. D-PoS begins to decrease slightly and our R-PoS system shows a strong linear decrease with more iterations.

\section{Limitations}
There were a few limitations our simulations faced. The first challenge was that we simulated the results of each of the various algorithms in isolated networks. In a real-world environment, there are thousands of different cryptocurrencies, each with their own network, and a node is free to convert their coins from one cryptocurrency to another. A node can then participate in the selection process within different networks, potentially making more money, and later  convert the coins back to the original cryptocurrency. The second limitation was that we focused on simulations on the incentive layer only. We did not analyze\slash model any other layers, so the results might be acceptable for the current layer, but they might negatively affect other layers. Thus, one layer might perform better within the simulations, but might hurt the overall blockchain overall if more layers were taken into account.

We also faced additional limitations within specific implementations. In D-PoS, our implementation allowed anyone to become a potential validator. However, in typical D-PoS, there are certain criteria a node must pass in order to become a delegate. This could impact centralization as it might eliminate many of the nodes in the network as potential delegates, thus leading to an increase in centralization over time.

In E-PoS, there were two main limitations we faced. The first limitation was determining who would validate a block if no one placed a bid for it. In the research paper, it mentioned that the total transaction cost of the block would decrease to enable the sender\slash receiver to validate it. Instead of this, we designed the system to select a node who placed a bid for a higher total transaction cost block. Additionally, in the case when there were no valid candidates (such as the first block), a random node was chosen from the network. The second limitation was determining the total transaction cost of the blocks. The initial E-PoS algorithm mentions that the number of potential validators should be roughly equal to the number of nodes in the network. This means that the total transaction cost should not exceed a majority of all the nodes' wealth and allow almost anyone to become a validator. However, since the main selection portion of the algorithm is to find validators with less blocks validated, this would introduce a potential sybil attack vulnerability as old validators can simply create new accounts (thus appearing to have 0 blocks validated) and have a better chance of being selected as a validator.

For R-PoS, it is extremely difficult to model human risk tolerance. We used a model that assumes that people will become greedier the more they keep winning and less greedy when they keep losing. However, this does not mean that someone on either end will continue to remain on the same decision pattern. It was even more difficult predicting a node's risk level when they laid in the middle of the greed spectrum because the issue was not only whether they stake or not, but also how many raffle tickets they would choose to buy.

\section{Future Work}
Overall, this process of creating simulations led to a whole new series of questions that could be answered with more work. The first thing that we would note in our simulations is that all of the networks exist in isolation. In the real world, network participants in one network have the ability to sell their tokens (and possibly incur some fee), and then allocate their resources to another token. In this regard, there may be some interactions between different tokens that we have not explored. If we had more time we would have been able to test integrating various combinations of Proof of Stake systems.

There was also some experimentation that we could have done with the simulation parameters. We had a fixed 10,000 participants in each simulation. We could have seen the results as the number of participants changes over time. We also have the additional complexity of non-staking transactions that wasn't completely encoded in the simulations. We assumed that there were some random variables representing the transaction fees, but we didn't model the transfer of wealth aside from the staking rewards. In doing so, we may find different results if the model ends up self re-balancing. We would also like to research further into different initializations of wealth distribution and block rewards.

On top of the future work regarding the overall simulation, there are some things that each individual model could also have explored. On the P-POS front, we decided that our algorithm will select 5 validators, which assumes only 5 blocks needed to be validated each round. However, maybe if there are more\slash less blocks, the number of validators would change and centralization may be impacted. For D-POS, we have constructed the simulation such that all network participants could be mining pool operators. This however is infeasible in practice because of the effort, technical skill, and resources needed to host a full validator. Thus, it would have been better to model the network with these costs accounted for. For the R-PoS and D-POS models we also could have implemented a bit more formal Nash equilibrium game theoretic decisions based on discounted future rewards that may have led to different playing strategies.
Overall, this research has opened up the possibility of exploring many aspects of Proof-of-Stake and the role it plays on centralization.
