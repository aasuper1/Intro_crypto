\chapter{Introduction}
Upon the creation of Bitcoin in 2009, cryptocurrencies painted an idealistic vision of decentralization. They promised a form of digital currency without the influence of a central authority, open for everyone to participate, and distributing decision-making among all of its users. But, why were such features desirable? The issue with centralized currency is that significant trust is placed on a single entity, which ends up holding a disproportionate amount of power over the rest of the population. If that entity tries to break or influence the rules placed upon it, everyone else in the network can be negatively impacted. Taking a look at a standard fiat currency like the US dollar, it derives its value based on people's trust in a central authority, the US Government. The dollar is printed by the US Treasury Department and all transactions are managed by banks. However, a bill can be forged and a bank transaction can be hacked.

Decentralization through cryptocurrencies was designed as a solution to that problem. Bitcoin utilizes blockchain technology to provide a peer-to-peer platform for digital currency in a trustless network. The blockchain is essentially a public ledger of transactions shared across all the users in the network. With all transactions out in the open, there is no need for a central entity to issue currencies or verify transactions.   Decentralization ensures there is one single agreed upon standard despite the existence of malicious users in a trustless environment. Due to these features, the realm of cryptocurrencies has snowballed in popularity. The industry as a whole has grown into a multi-trillion dollar market.

Despite the promises of decentralization, people have found a way to work around those constraints. With the rising popularity, centralized behavior has emerged as an increasingly serious concern. Most cryptocurrencies have started to demonstrate more centralized behavior in their infrastructure in the form of mining farms, cryptocurrency exchanges, and mining pools. Malicious users can use these centralized behaviors to gain some form of control over the network and influence the price of the coin to their favor \cite{CRYPTO:1}. By forming groups or increasing their hashing power, they have, in essence, become a "single" entity that controls the network by determining which blocks get validated. If a mining pool grows too powerful, the group has the potential to initiate a 51\% attack on the network to enable double-spending. There have also been cases of powerful mining pools exhibiting cartel behavior by using their collective wealth to threaten the blockchain.

Hashing power isn't the only area facing issues of centralized behavior; we can see it happen across many layers. Geographically, of the major Bitcoin mining pools, around 76\% are located in China where electricity is cheaper \cite{CRYPTO:2}. Similarly, in the hardware layer, the mining industry is dominated by ASICs (custom hardware designed specifically for mining) and have been increasingly monopolized by Chinese miners. On the economic layer, the emergence of centralized exchanges like Coinbase and Binance, and yield-farming have created large capital centralization. On the governance layer, governance tokens based on voting power without economic value are centralized in a few well-known community members, such as Linda Xie having 30\% of voting tokens for Gitcoin. Through these observations, we can observe that blockchain solutions have become a lot more centralized than originally envisioned. Thus, the fundamental concept of decentralization most cryptocurrencies strive for has become compromised and new solutions need to be designed.

The purpose of our investigation is to devise some system or protocol for cryptocurrencies that can reduce the amount of centralized behavior that occurs. However, there are a massive number of factors that influence the blockchain's tendency towards centralized behavior. Because of this, it is challenging to narrow down the scope of the problem to find the solution. It is extremely difficult to design a metric around a blockchain's "centralization" since it comes from different sources such as the geographical distribution of users, consensus centralization (mining pools), wealth concentration (centralized exchanges), and even sources outside of the scope of the blockchain, like hardware availability (ASICs) \cite{CRYPTO:5}. Therefore, because there are so many different classifications of centralization across many layers of operations, we need to define which specific centralization problem our investigation will attempt to solve.

As many investigations have already been conducted on centralized hashing power in Proof-of-Work currencies and because Proof-of-Stake (PoS) is becoming more popular and still evolving, we decided to steer our investigation towards centralization in Proof-of-Stake. Of the various aspects of centralization occurring in PoS currencies, the one we believe to be the most pressing issue is wealth centralization. Just like shares in a company, PoS gives greater power to those with more coins. To make things worse, higher stake can influence future income, causing the rich to get richer and further exacerbating the centralization problem. In this report, we will be investigating if wealth centralization tends to occur in PoS systems, focusing only on the economic layer. We will look at different PoS protocols to observe what factors lead to an increase or decrease in centralization. Then, we will perform agent-based modeling across these protocols and compare the effects on wealth distribution across the network. With the results, we will look at the incentive layer to try to use a game theoretic approach and develop our own PoS protocol in an effort to minimize wealth centralization.

