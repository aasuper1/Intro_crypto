%%% File encoding is ISO-8859-1 (also known as Latin-1)
%%% You can use special characters just like �,� and �

% ##############################################
% Start: Table of Contents (TOC) Customization
% ##############################################
%

% Level for numbered captions
\setcounter{secnumdepth}{5}

% Level of chapters that appear in Table of Contents
\setcounter{tocdepth}{5} % bis wohin ins Inhaltsverzeichnis aufnehmen
% -2 no caption at all
% -1 part
% 0  chapter
% 1  section    
% 2  subsection 
% 3  subsubsection
% 4  paragraph
% 5  subparagraph

% KOMA-Script code to adjust TOC
% Applying the color 'myColorMainA' which is defined in the main file (MainFile.tex)
\makeatletter
\addtokomafont{chapterentrypagenumber}{\color{myColorMainA}}
\addtokomafont{chapterentry}{\color{myColorMainA}}
\makeatother

%
% #######################
% End: Table of Contents (TOC) Customization
% #######################

% ##############################################
% Start: Floating Object Customization
% ##############################################
%

% Extended support for catioons of figures and tables etc.
% CTAN: http://www.ctan.org/pkg/caption
\usepackage[%
	font={small},
	labelfont={bf,sf},
	format=hang, % try plain or hang
	margin=5pt,
]{caption}
%

% #######################
% End: Floating Object Customization
% #######################

% ##############################################
% Start: Headings Customization
% ##############################################
%

% KOMA-Script code to customize the headings
% Applying the color 'myColorMainA' which is defined in the main file (MainFile.tex)
\addtokomafont{chapter}{\color{myColorMainA}}
\addtokomafont{section}{\color{myColorMainA}}
\addtokomafont{subsection}{\color{myColorMainA}}
\addtokomafont{subsubsection}{\color{myColorMainA}}
\addtokomafont{paragraph}{\color{myColorMainA}}
\addtokomafont{subparagraph}{\color{myColorMainA}}

% #######################
% End: Headings Customization
% #######################
